\newpage
\section{Gott legt seinen Geist auf jung und alt}
\index{Glanz!Jesu}

\textbf{\textit{E(t,e) | Jahwe | Geist}}
\index{Externalisierung E(t,e)}
\index{Jahwe}
\index{Geist}

\biblerefformat{kurz}
\bibleverse{Num}(11:16-17.25-30)
\begin{BibelSt}
Da sprach der HERR zu Mose: Versammle mir siebzig von den Ältesten Israels, die du kennst, weil sie die Ältesten des Volkes und seine Listenführer sind; bring sie zum Offenbarungszelt! Dort sollen sie mit dir zusammen hintreten. Dann komme ich herab und rede dort mit dir. Ich nehme etwas von dem Geist, der auf dir ruht, und lege ihn auf sie. So können sie mit dir zusammen an der Last des Volkes tragen und du musst sie nicht mehr allein tragen. \!
Der HERR kam in der Wolke herab und redete mit Mose. Er nahm etwas von dem Geist, der auf ihm ruhte, und legte ihn auf die siebzig Ältesten. Sobald der Geist auf ihnen ruhte, redeten sie prophetisch. Danach aber nicht mehr.
Zwei Männer aber waren im Lager geblieben; der eine hieß Eldad, der andere Medad. Auch über sie kam der Geist. Sie gehörten zu den Aufgezeichneten, waren aber nicht zum Offenbarungszelt hinausgegangen. Auch sie redeten prophetisch im Lager. Ein junger Mann lief zu Mose und berichtete ihm: Eldad und Medad sind im Lager zu Propheten geworden. Da ergriff Josua, der Sohn Nuns, der von Jugend an der Diener des Mose gewesen war, das Wort und sagte: Mose, mein Herr, hindere sie daran! Doch Mose sagte zu ihm: Willst du dich für mich ereifern? Wenn nur das ganze Volk des HERRN zu Propheten würde, wenn nur der HERR seinen Geist auf sie alle legte! Dann zog sich Mose mit den Ältesten Israels in das Lager zurück.
\end{BibelSt}

\subsection{Impuls}
\begin{impuls}

\begin{description}
\item[]

\end{description}

\end{impuls}

\subsection{Heiliger Geist}
\cite{KHH} vom 5.6.
\begin{gedicht}
\begin{verse}
er kommt\\
   kommt\\
   kommt\\
durchkommt er\\
durch die zeit\\
hebt sie auf\\
wird gegenwart\\!

erwartet\\
uns\\
in der zukunft\\!
\end{verse}
\end{gedicht}
\subsection{Lied: Veni Creator, litanie (13)}
Veni Creator spiritus.

\subsection{Gebet}
\footnote{\cite{FR-heute} vom }
