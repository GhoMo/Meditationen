
%\chapterimage{chapter_head_2.pdf} % Chapter heading image

\chapter{3x4 Meditationen}
\newpage
\section{Der Glanz auf Moses Gesicht}
\index{Glanz!Mose}

\textbf{\textit{S(t,t) | Gott, Jahwe | Spiegelung}}
\index{Sozialisierung S(t,t)}
\index{Gott!Jahwe}
\index{Spiegelung}

\biblerefformat{kurz}
\bibleverse{Ex}(34:27-35)
\begin{BibelSt}
Dann sprach der Herr zu Mose: Schreib diese Worte auf! Denn aufgrund dieser Worte schließe ich mit dir und mit Israel einen Bund. Mose blieb dort beim Herrn vierzig Tage und vierzig Nächte. Er aß kein Brot und trank kein Wasser. Er schrieb die Worte des Bundes, die zehn Worte, auf Tafeln. Als Mose vom Sinai herunterstieg, hatte er die beiden Tafeln der Bundesurkunde in der Hand. Während Mose vom Berg herunterstieg, wusste er nicht, dass die Haut seines Gesichtes Licht ausstrahlte, weil er mit dem Herrn geredet hatte. Als Aaron und alle Israeliten Mose sahen, strahlte die Haut seines Gesichtes Licht aus und sie fürchteten sich, in seine Nähe zu kommen. Erst als Mose sie rief, kamen Aaron und alle Sippenhäupter der Gemeinde zu ihm zurück und Mose redete mit ihnen. Dann kamen alle Israeliten herbei und er übergab ihnen alle Gebote, die der Herr ihm auf dem Sinai mitgeteilt hatte.  Als Mose aufhörte, mit ihnen zu reden, legte er über sein Gesicht einen Schleier.  Wenn Mose zum Herrn hineinging, um mit ihm zu reden, nahm er den Schleier ab, bis er wieder herauskam. Wenn er herauskam, trug er den Israeliten alles vor, was ihm aufgetragen worden war.  Wenn die Israeliten das Gesicht des Mose sahen und merkten, dass die Haut seines Gesichtes Licht ausstrahlte, legte er den Schleier über sein Gesicht, bis er wieder hineinging, um mit dem Herrn zu reden.
\end{BibelSt}
\subsection{Impuls}
\begin{impuls}
\begin{description}
\item[Gespiegeltes Wissen.] "Gespiegeltes Wissen ist kein 'logisches Wissen', sondern reflektiertes und empfangenes Wissen. Deshalb ist es so schwierig, einem Menschen Gott … oder die Liebe zu beweisen, wenn er nicht selbst schon Empfänger geworden ist. Tatsächlich ist es fast unmöglich. Erinnern Sie sich, wie Moses Gesicht leuchtete, als er den Blick Gottes empfangen hatte und mit Wahrhaftigkeit und Liebe angesehen worden war? Und doch verschleierte er sein Gesicht, als er seinem Volk gegenübertrat. Das ist ein grosses Symbol. Alle Menschen müssen als sie selbst und für sie selbst angesehen werden und den Blick Gottes ganz persönlich empfangen. Es reicht nicht, sich auf das Sehen oder Gesehenwerden eines anderen zu verlassen."\footnote{\cite[48]{Tanz}}
\item[Auszeiten und Orte der Begegnung.] Mose ist unterwegs in eine ungewisse Zukunft. Er hat die Verantwortung für ein ganzes Volk. Der Berg Sinai ist sein Ort der Begegnung mit Gott. Dieser Ort ist Mose vorbehalten, niemand sonst darf den Berg besteigen. 
\footnote{
\biblerefformat{kurz}
\bibleverse{Ex}(19:12)
}
Moses nimmt sich Zeit, zieht sich für 40 Tage zurück. Dabei ist dies schon das zweite Mal, dass Moses Gott auf dem Sinai sucht und findet. während der ersten Begegnung führte die Abwesenheit von Moses dazu, dass die Israeliten das Goldene Kalb fertigten.
\footnote{
\biblerefformat{kurz}
\bibleverse{Ex}(32:)
}
Trotzdem nimmt sich Moses eine zweite "Auszeit" und zieht sich für längere Zeit zurück. Während der Zeit zählt nur die Begegnung mit Gott, selbst auf die Gefahr hin dass jene, die ihm anvertraut sind, in seiner Abwesenheit wieder vom Weg abkommen und anderen Götzen nachlaufen. 
\item[Tafeln — den Bund fassbar machen.] Die ersten Tafeln, welche Moses aus Enttäuschung über das Goldene Kalb zerstört hatte, waren von Gott selber geschrieben. 
\footnote{
\biblerefformat{kurz}
\bibleverse{Ex}(31:18),
\biblerefformat{kurz}
\bibleverse{Ex}(32:19)
}
Nun ist Moses wieder auf dem Sinai, und diesmal schreibt er die Worte des Bundes selber auf die Tafeln. Diese Tafeln sind eine Hilfe, um die Gegenwart des unsichtbaren Gottes fassbar, greifbar zu machen. Die Mesusa, die Schriftrolle am Türpfosten, soll Gottes Wort und seine Gegenwart auch im Alltag sichtbar und greifbar machen.

\item[Ein unmerklicher Prozess.] Moses konzentriert sich auf die Kommunikation. Er empfängt von Gott die Worte, die er auf die Tafeln schreibt, und er gibt diese Worte an die Israeliten weiter. In dem Zusammensein mit Gott passiert jedoch eine Veränderung, dessen er sich selber nicht bewusst ist. Er erfährt eine "Erleuchtung", ein Glanz, der von ihm ausstrahlt. Als er dem Volk die Tafeln und die Worte Gottes überbringt, wirken nicht nur seine Worte, sondern auch seine Ausstrahlung, auch wenn er sie zu verschleiern versucht.
\end{description}

\begin{itemize}
\texitit{
    \item Ich darf mir längere Auszeiten nehmen. 
    \item Gibt es einen Ort an dem ich mich Gott nahe fühle? Wo ist mein Ort, wo ich mich zurückziehen und wohin ich immer wieder zurückkehren möchte?
    \item Wo ist mein Ort der Begegnung mit Gott im Alltag?
    }
\end{itemize}

\end{impuls}
\subsection{du spiegelst mein gesicht}
\cite{KHH} vom 8.9.
\begin{gedicht}
\begin{verse}
weiher meiner kindheit\\
versteckt im wald\\
umstanden von riedgrass\\
über das goldene käfer laufen\\
du spiegelst mein gesicht\\
und nebelstreifen\\
schleifen darüber\\!
mir wird schwindlig vor freude\\
weil du da bist\\
su-su-sirren die libellen\\
kräuselt das wasser\\
und du bist noch da\\
und er ist noch da\\
der ewige-eine\\
dessen gesicht mich\\
aus der tiefe anschaut
\end{verse}
\end{gedicht}
\subsection{Lied: Qui regarde vers Dieu (65/66)}
\begin{lied}
Qui regarde vers Dieu resplendira sur son visage,\\plus d’amertume sur son visage, plus d'amertume sur son visage.
\biblerefformat{kurz}
\bibleverse{Ps}(34:6)
\end{lied}

\subsection{Gebet}
Wer geduldig ist und in einen unerlässlichen Reifungsprozess einwilligt, erlebt den Tag, an dem sich sein inneres Wesen auferbaut hat – ohne dass er darum wusste.\footnote{\cite{FR-heute} vom 12.1.}