\newpage
\section{Die Vision der Auferweckung Israels}
\index{Geist!Leben}
\index{Auferstehung!Auferweckung}

\textbf{\textit{I(e,t) | Jahwe | Geist}}
\index{Internalisierung I(e,t)}

\biblerefformat{kurz}
\bibleverse{Ezechiel}(37:1-14)
\begin{BibelSt}
Die Hand des HERRN legte sich auf mich und er brachte mich im Geist des HERRN hinaus und versetzte mich mitten in die Ebene. Sie war voll von Gebeinen. Er führte mich ringsum an ihnen vorüber und siehe, es waren sehr viele über die Ebene hin; und siehe, sie waren ganz ausgetrocknet. Er fragte mich: Menschensohn, können diese Gebeine wieder lebendig werden? Ich antwortete: GOTT und Herr, du weißt es. Da sagte er zu mir: Sprich als Prophet über diese Gebeine und sag zu ihnen: Ihr ausgetrockneten Gebeine, hört das Wort des HERRN! So spricht GOTT, der Herr, zu diesen Gebeinen: Siehe, ich selbst bringe Geist in euch, dann werdet ihr lebendig. Ich gebe euch Sehnen, umgebe euch mit Fleisch und überziehe euch mit Haut; ich gebe Geist in euch, sodass ihr lebendig werdet. Dann werdet ihr erkennen, dass ich der HERR bin. Da sprach ich als Prophet, wie mir befohlen war; und noch während ich prophetisch redete, war da ein Geräusch: Und siehe, ein Beben: Die Gebeine rückten zusammen, Bein an Bein. Und als ich hinsah, siehe, da waren Sehnen auf ihnen, Fleisch umgab sie und Haut überzog sie von oben. Aber es war kein Geist in ihnen. Da sagte er zu mir: Rede als Prophet zum Geist, rede prophetisch, Menschensohn, sag zum Geist: So spricht GOTT, der Herr: Geist, komm herbei von den vier Winden! Hauch diese Erschlagenen an, damit sie lebendig werden! Da sprach ich als Prophet, wie er mir befohlen hatte, und es kam der Geist in sie. Sie wurden lebendig und sie stellten sich auf ihre Füße - ein großes, gewaltiges Heer.
Er sagte zu mir: Menschensohn, diese Gebeine sind das ganze Haus Israel. Siehe, sie sagen: Ausgetrocknet sind unsere Gebeine, unsere Hoffnung ist untergegangen, wir sind abgeschnitten. Deshalb tritt als Prophet auf und sag zu ihnen: So spricht GOTT, der Herr: Siehe, ich öffne eure Gräber und hole euch, mein Volk, aus euren Gräbern herauf. Ich bringe euch zum Ackerboden Israels. Und ihr werdet erkennen, dass ich der HERR bin, wenn ich eure Gräber öffne und euch, mein Volk, aus euren Gräbern heraufhole. Ich gebe meinen Geist in euch, dann werdet ihr lebendig und ich versetze euch wieder auf euren Ackerboden. Dann werdet ihr erkennen, dass ich der HERR bin. Ich habe gesprochen und ich führe es aus - Spruch des HERRN.
\end{BibelSt}

\subsection{Impuls}
\begin{impuls}
\begin{description}
\item[Wirken des Geistes] «Der Heilige Geist … lässt uns wachsen und hält uns verletzlich. Fur das Leben und die Liebe. Bezeichnend dabei ist, dass die wichtigsten biblischen Metaphern für den Heiligen Geist immer dynamisch, energisch, und beweglich sind.»\footnote{\cite{Tanz} S. 54}
\item[Prophet] Diese Bibelstelle zeigt die Verinnerlichung (Internalisierung) auf unterschiedlichen Ebenen. Als Teil der Vision sieht der Prophet wie die Kraft Gottes den ausgetrockneten Gebeinen wieder Leben schenkt. In der Vision ist die Schaffung des Körpers aus Fleisch getrennt von der Kraft des Geistes, der den Menschen wieder Leben einhaucht. Die Verinnerlichung ist aber noch viel umfassender. Gott zeigt dem Propheten nicht nur seine Kraft, sondern auch, dass diese durch das Wort des Propheten zur Entfaltung kommt. Ezechiel überlässt Gott die Entscheidung, wo und wie er seine Macht zeigen möchte. Gott fragt: «Ist es möglich?» und Ezechiel antwortet voll Vertrauen «GOTT und Herr, du weisst es.» Gott beschränkt sich nicht darauf, das seine Kraft und das Wirken des Geistes zu demonstrieren. Er fordert den Propheten bewusst auf: Sprich Du zu den Gebeinen; rufe Du den Geist aus den vier Winden herbei. In der Vision lässt Gott den Propheten seine Aufgabe als Vermittler erfahren und erlernen.
\item[Ackerboden] Aus der Perspektive der Menschen, die alle Hoffnung verloren haben (die ausgetrockneten Gebeine), sind es die Worte des Propheten, die Gottes Wirken ankündigen und real werden lassen. Als Teil dieses neuen Lebens werden sie auf ihren Ackerboden zurückgebracht. Sie werden wieder Teil des Kreislaufs des Lebens und der Natur.
\end{description}
\end{impuls}

\subsection{ich hüte das wort}
\cite{KHH} vom 4.6
\begin{gedicht}
\begin{verse}
frau unter den sternen\\
wer bist du?\\!
ich hüte das wort\\
unterm kelch\\
und achte auf\\
das brausen\\
des geistes\\
in den lüften
\end{verse}
\end{gedicht}

\subsection{Der Prophet — der Abschied}
«Ich drücke nur in Worten für euch aus, was ihr in Gedanken selber wisst. Und was ist Wissen in Worten anderes als ein Schatten wortlosen Wissens? Eure Gedanken und meine Worte sind Wellen aus einem versiegelten Gedächtnis, das Bericht gibt von unseren gestrigen Tagen, und von den alten Tagen, da die Erde weder uns noch sich selber kannte, und von den Nächten, da die Erde in Verwirrung aufgewühlt war. Weise sind zu euch gekommen, um euch von ihrer Weisheit zu geben. Ich kam, um von eurer Weisheit zu nehmen: Und, seht, ich habe gefunden, was größer ist als Weisheit. Es ist ein Flammengeist in euch, der sich immer mehr steigert, während ihr, seiner Entfaltung ungeachtet, das Vergehen eurer Tage beklagt. Es ist das Leben auf der Suche nach Leben in Körpern, die vor dem Grab erzittern.
Hier gibt es keine Gräber. Diese Berge und Ebenen sind eine Wiege und ein Trittstein. Wann immer ihr an dem Feld vorbeikommt, in dem ihr eure Vorfahren beigesetzt habt, schaut richtig hin, und ihr werdet euch und eure Kinder Hand in Hand tanzen sehen.»\footnote{\cite{prophet}S. 116}

\subsection{Lied: I am sure I shall see (127)}
I am sure I shall see the goodness of the Lord in the land of the living. Yes, I shall see the goodness of our God, hold firm, trust in the Lord. \biblerefformat{kurz}
\bibleverse{Ps}(27:13)

\subsection{Gebet}
Scheinbar entbehrt die endlose Wiederholung der gleichen Worte im Gebet jeder Spontaneität. Aber nach langem Warten brechen unversehens innere Quellen auf, eine Fülle, die Gegenwart des Heiligen Geistes, die immer wieder aufrüttelt.
\footnote{\cite{FR-heute} vom 16.5.}
