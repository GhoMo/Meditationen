
\section{Trostworte an die Jünger}
\index{Geist!Erinnerung}

\textbf{\textit{C(e,e) | Geist | Erinnerung}}
\index{Kommunikation C(e,e)}

\biblerefformat{kurz}
\bibleverse{Joh}(14:15-26)
\begin{BibelSt}
Wenn ihr mich liebt, werdet ihr meine Gebote halten. Und ich werde den Vater bitten und er wird euch einen anderen Beistand geben, der für immer bei euch bleiben soll, den Geist der Wahrheit, den die Welt nicht empfangen kann, weil sie ihn nicht sieht und nicht kennt. Ihr aber kennt ihn, weil er bei euch bleibt und in euch sein wird. Ich werde euch nicht als Waisen zurücklassen, ich komme zu euch. Nur noch kurze Zeit und die Welt sieht mich nicht mehr; ihr aber seht mich, weil ich lebe und auch ihr leben werdet. An jenem Tag werdet ihr erkennen: Ich bin in meinem Vater, ihr seid in mir und ich bin in euch. Wer meine Gebote hat und sie hält, der ist es, der mich liebt; wer mich aber liebt, wird von meinem Vater geliebt werden und auch ich werde ihn lieben und mich ihm offenbaren. Judas - nicht der Iskariot - fragte ihn: Herr, wie kommt es, dass du dich nur uns offenbaren willst und nicht der Welt? Jesus antwortete ihm: Wenn jemand mich liebt, wird er mein Wort halten; mein Vater wird ihn lieben und wir werden zu ihm kommen und bei ihm Wohnung nehmen. Wer mich nicht liebt, hält meine Worte nicht. Und das Wort, das ihr hört, stammt nicht von mir, sondern vom Vater, der mich gesandt hat.
Das habe ich zu euch gesagt, während ich noch bei euch bin. Der Beistand aber, der Heilige Geist, den der Vater in meinem Namen senden wird, der wird euch alles lehren und euch an alles erinnern, was ich euch gesagt habe.
\end{BibelSt}

\subsection{Impuls}
\begin{impuls}
\begin{description}
\item[Erinnerung]«Wir haben versucht, Gott durch objektiviertes Wissen kennenzulernen, und das Ergebnis ist eine langweilige Schwarzweiss-Kopie, weil wir nicht selbst beteiligt sind. … Tatsache ist aber: Menschen begeistern sich nur für etwas, was sie in irgendeiner Weise einschliesst. Das weiss Gott, und deshalb schliesst er uns in sein eigenes Wissen ein – in dem er den Heiligen Geist in uns hineingibt, den inneren Wissenden, der uns an alles erinnert. Erinnern kann man sich nur an etwas, was man schon einmal im Inneren wusste. Eine ganz andere Form von Begreifen, Verinnerlichen, wird uns hier gegeben.»\footnote{\cite{Tanz}, Seite 49f. mit Verweis auf 
\biblerefformat{kurz}
\bibleverse{Joh}(14:26)}
\item[Dreiklang-Einklang] «Ich bin in meinem Vater, ihr seid in mir und ich bin in euch» und der Geist der Wahrheit ist auch in den Jüngern, in mir, in allen, die durch das Gebot der Liebe vereint sind. Ich lausche den Schwingungen der Klangschale, die mich an diesen Einklang erinnert.
\item[den Auferstandenen sehen] Jesus erklärt den Jüngern, dass sie ihn sehen werden, selbst wenn er für die Welt nicht mehr sichtbar ist. Können Petrus, Jakobus und Johannes, die Zeugen der Verklärung, welche auch Moses und Elia sehen konnten, diese Worte besser verstehen? Die Erscheinung von Moses und Elia war eine erste Begegnung mit Menschen, die eigentlich schon lange tot sind («Externalisierung»). Der nächste Schritt in der Lernspirale besteht darin, diese Erfahrung mit den anderen Elementen zu kombinieren: Die Worte Jesu, das Gebot der Liebe, die ständige Gegenwart und die Einheit von Gott, Jesus und Geist. Auch wenn die Jünger den Zusammenhang noch nicht erfassen können, so kündigt ihnen Jesus an, dass die Kraft des Geistes später ermöglichen wird, diese Wahrheiten zu verinnerlichen.
\item[Liebe] Die Liebe ist der Schlüssel zur Einheit mit dem dreieinigen Gott. Wer liebt, erfüllt das Gebot, das zur Einheit mit Gott führt. Doch kann man liebe «gebieten»? Kann man Liebe definieren, beschreiben? Es ist der Geist in mir, der mich berührt, mich zu Tränen rührt und daran erinnert, was Liebe im Herzen bedeutet.
\end{description}
\end{impuls}

\subsection{Heiliger Geist - wer bist du?}
\cite{KHH} vom 22.5.
\begin{gedicht}
\begin{verse}
du\\
Heiliger Geist\\
wer bist du?\\
wind, flamme, gesang?\\
bist du das licht, das\\
alles in sich zusammenhält?\\
die wirklichkeit Gottes?\\
seine zärtlichkeit?\\
bist du die tiefe\\
innerlichkeit eines\\
menschen, der lächelt\\
weil er sich von\\
Gott angesehen weiss?
\end{verse}
\end{gedicht}
\subsection{Lied: Esprit consolateur (126)}
Esprit consolateur, amour de tout amour.

\subsection{Gebet}
Wissen, wo unser Herz ausruhen kann, heisst eine Wirklichkeit erfassen, die unseren Augen verborgen ist: Christus begleitet uns. Unser bedrücktes Herz lebt von neuem auf. Es beginnt, mitunter lautlos, zu singen: Der Atem deiner Liebe hat mich durchströmt, ich trete nicht auf der Stelle, ich gehe mit dir.\footnote{\cite{FR-heute} vom 7.2.}


