\newpage
\section{Die Heilung einer blutflüssigen Frau}
\index{Heilung}

\textbf{\textit{S(t,t) | Jesus | Berührbarkeit}}
\index{Sozialisierung S(t,t)}
\index{Jesus}
\index{Berührbarkeit}


\biblerefformat{kurz}
\bibleverse{Mt}(9:20-22)
\begin{BibelSt}
Und siehe, eine Frau, die schon zwölf Jahre an Blutfluss litt, trat von hinten heran und berührte den Saum seines Gewandes; denn sie sagte sich: Wenn ich auch nur sein Gewand berühre, werde ich geheilt. Jesus wandte sich um, und als er sie sah, sagte er: Hab keine Angst, meine Tochter, dein Glaube hat dich gerettet! Und von dieser Stunde an war die Frau geheilt. 
\end{BibelSt}

\subsection{Impuls}
\begin{impuls}
\begin{description}
\item[Isolierte Frau] Die Periode ist ein Teil des Kreislaufes der Natur, der den weiblichen Körper ständig verändert, so wie die Jahreszeiten. Die Natur der Frau ist Teil des Kreislaufes des Lebens, der Schöpfung; doch nach dem Gesetz macht dieser Blutfluss unrein. Wer eine Frau während ihrer Periode berührt, wird dadurch selber unrein \biblerefformat{kurz}\bibleverse{Lev}(15:19). Durch dieses Gesetz ist die Frau nun schon seit 12 Jahren isoliert. Jesus zu berühren ist aus der Sicht des Gesetzes ein grosses Vergehen, weil dadurch er dadurch selber unrein wird. Indem die Frau die Nähe zu Jesus sucht und aus ihrem Glauben heraus die Grenzen des Gesetzes überwindet, entsteht eine Beziehung, in der die Kraft Gottes wirkt. Gott möchte unser Heil und unsere Heilung.

\item[Berührbarkeit] «Sie haben sicher schon bemerkt, dass Jesus keine Checkliste abhakt, bevor er jemanden heilt. Er fragt nur: "Lässt du zu, dass ich dich berühre? Wenn ja, dann kann es losgehen." Die Berührbaren werden geheilt. So einfach ist das. … Es gibt nur eine Frage: \textit{Willst du geheilt werden?} Wenn die Antwort verletzlich, vertrauensvoll oder zuversichtlich ist, dann fliesst der Fluss und der Mensch wird geheilt.» \footnote{\cite{Tanz} S. 55}
In dieser Bibelstelle geschieht dieser Fluss der Kraft ganz ohne Worte. 
\footnote{Bei \biblerefformat{kurz}\bibleverse{Mk}(5:21-43) und \bibleverse{Lk}(8:43-48) erklärt die Frau erst den Umstehenden, warum sie es gewagt hat Jesus zu berühren, und erklärt, dass diese Berührung sie tatsächlich geheilt hat.}
Es genügt die Berührung, und als Jesus die Frau ansieht, weiss er um ihre Not und ihren Glauben.

\item[Ausbluten] Der gesunde Zyklus ist ein Wechsel von Ausfluss und Sammlung. Wenn dieser Zyklus gestört ist, und der Mensch nur noch ausblutet, leidet er/sie unter dem ständigen Verlust an Lebenskraft und Lebensfreude. Dies gilt auch im übertragenen Sinne. Wenn ich das Gefühl habe auszubluten, brauche ich die Berührung Gottes, um den Kreislauf des Verschenkens und der Sammlung wieder ins Gleichgewicht zu bringen.
\end{description}

%\begin{itemize}
%\texitit{
%\item 
   
%    }
%\end{itemize}

\end{impuls}
\subsection{Rühre mich an}

\begin{gedicht}
\begin{verse}
Ich möchte dich berühren, Herr\\
und wenn es nur der Saum deines Gewandes ist,\\
den ich halten kann\\!

Ich möchte dich berühren, Herr\\
und wenn es nur ein Wort deiner Botschaft ist,\\
die ich fassen kann.\\!

Ich möchte dich berühren, Herr,\\
möchte mich heran-tasten an dich.\\!

Ich möchte dich berühren, Herr,\\
deinen Saum,\\
deinen Finger,\\
dein Wort.\\!

Ich möchte dich berühren, Herr,\\
und ahnen dein Gewand,\\
deine Hand,\\
deine Botschaft.\\!

Ich möchte dich berühren, Herr,\\
und fühlen die Kraft, die ausströmt,\\
die Wärme, die belebt,\\
das Leben, das heilt.\\!

Ich möchte dich berühren, Herr,\\
und ich wage es,\\!

Ich rühre dich an\\
Rühre du mich an, Herr,\\
fasse mich,\\
ergreife mich,\\
halte mich –\\
heile mich.\\
\poemauthorcenter{Marie-Lousie Langwald\footnote{\cite{biblFrauen} S. 40}}

\end{verse}
\end{gedicht}
\subsection{Lied: Voici Dieu qui vient à mon secours (142)}
\begin{lied}
Voici Dieu qui vient à mon secours, le Seigneur avec ceux qui me soutiennent. Je te chante, toi qui me relèves. Je te chante, toi qui me relèves. \biblerefformat{kurz}
\bibleverse{Ps}(54:6) und \bibleverse{Ps}(30:2)
\end{lied}

\subsection{Gebet}
Es wagen, sich über alles zu freuen, was Gott in uns und um uns vollbringt. Und alles, wofür man bei sich selbst und bei den anderen schwarz sieht, alles, was einem den Seelenfrieden raubt, löst sich auf.\footnote{\cite{FR-heute} vom 29.4}